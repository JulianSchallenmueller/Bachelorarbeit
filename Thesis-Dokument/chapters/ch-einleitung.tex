\begin{spacing}{1.2}

\chapter{Einleitung}
\label{sec:einl}

\section{Einleitung}
Eines der wichtigsten Schlagworte im Zeitalter der fortschreitenden
Digitalisierung ist der Begriff des Cloud Computings. Auch in Bereichen
der Industrie wie der Finanz-
und Versicherungsbranche die sich zu großen Teilen aufgrund von
Sicherheits- und anderen Bedenken lange Zeit gegen die Nutzung
Cloud-basierter Systeme entschieden hatte gewinnt das Thema mehr und mehr
Relevanz. Die Nutzung von Cloud-Technologien verspricht die Möglichkeit
schneller auf Anforderungen von Kunden reagieren zu können,
kostengünstige und flexible Skalierung der eigenen Rechenkapazitäten,
Einsparungen durch den Wegfall eigener IT-Infrastruktur Fachleute und mehr.\\
Gemeinsam mit der Eröffnung neuer Möglichkeiten bringt die Einführung neuer
Technologie auch immer eine Reihe eigener Herausforderungen
mit sich. Für eine erfolgreiche und gewinnbringende Einführung dieser ist
es essentiell diese zu verstehen und die passende Denkweisen und Werkzeuge
mit denen die aufkommenden Probleme gelöst werden können entsprechend
einzusetzen.

\section{Motivation und Ziele der Arbeit}
Das Thema mit dem sich diese Arbeit befassen wird ist das automatisierte
Managen und Bereitstellen von Cloud Computing Resourcen,
zusammengefasst unter dem Begriff Infrastructure as Code (IaC). Die Grundlagenkapitel werden zu diesem Zweck auf
den technischen Kontext und die Relevanz von Cloud Computing, IaC und
dem Software Tool Terraform eingehen.
Es wird erläutert werden an welcher Stelle die ausgewählten Plattformen
und Software zum Einsatz kommen, welche Probleme dadurch gelöst werden,
wo deren Vorteile und Grenzen liegen sowie welche Alternativen existieren
und wo ergänzende Werkzeuge zum Einsatz kommen können.\\
Den Kern der Arbeit bildet ein Vergleich der ausgewählten Cloud
Service Provider auf Basis der Unterstzützung Von IaC mit Terraform.
Hierfür wird die Infrastruktur einer Schulung der Firma Novatec
Consulting GmbH in der die wichtigsten grundlegenden Cloud Computing
Resourcen Verwendung finden auf verschiedenen Plattformen mit Terraform
bereitgestellt und einem Vergleich unterzogen.
Als Vergleichsmetriken werden Kriterien der Norm ISO/IEC 25000
herangezogen um eine fachgerechte und neutrale Bewertung zu gewährleisten.
Dabei soll auch in Betracht gezogen werden wie hoch der Aufwand für die
Migration eines bestehenden Systems zu IaC ausfällt.
Nach der Auswertung der Vergleichsergebnisse werden die gewonnenen
Erkentnisse zusammengefasst, weitere untersuchenswerte Aspekte aufgeführt
und ein Ausblick auf aktuelle ud Zukünftige Entwicklungen gegeben. Durch
die sehr aktuelle Relevanz des Themas werden sich mit hoher
Wahrscheinlichkeit auch
in Zukunft neue Software und Technologien durchsetzen, alte verdrängt und
neue Lösungsansätze für bestehende Herausforderungen und Probleme
etabliert werden. Diese Arbeit wird jedoch nur die aktuell relevantentesten
PLattformen und Tools betrachten um eine Eintscheidungsbasis für deren
Einsatz zum aktuellen Zeitpunkt bieten zu können.

\end{spacing}