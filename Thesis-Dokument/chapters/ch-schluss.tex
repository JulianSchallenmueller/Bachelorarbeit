\begin{spacing}{1.2}

\chapter{Schluss}
\label{sec:schluss}

\section{Zusammenfassung der Arbeit}

Im Rahmen dieser Arbeit wurde ein Vergleich der Public Cloud Service
Provider Google Cloud Platform und Microsoft Azure in Bezug auf die
Unterstützung von Infrastructure as Code mit Terraform durchgeführt.
Um den Kontext, die technischen Grundlagen und die Motivation der Arbeit
darzustellen werden im ersten Abschnitt Funktionsprinzip, Vorteile und
Herausforderungen des modernen Cloud Computings sowie die Grundlagen
von Infrastructure as Code und Terraform zusammengefasst.
Grundlage für die Erklärung des Funktionsprinzip von Cloud Computing
stellt hierbei die weithin akzeptierte Definition des National
Institude of Standards and Technology der USA dar.
Zusätzlich werden verschiedene Vorteile und Herausforderungen moderner
Cloud Technologie kurz zusammengefasst und erläutert sowie ein
Überblick über die wichtigsten Cloud Service Provider im aktuellen
Markt hergestellt.

Die Grundlagen zu Infrastructure as Code werden durch den
technologischen Wandel und eine Erklärung des Cloud Age Mindset
eingeleitet. Die Notwendigkeit für den Einsatz von IaC wird durch
die bestehenden Probleme welche durch IaC gelöst werden dargestellt,
und drei Kernverfahren im Einsatz von IaC erläutert.
Zum Abschluss der Grundlagen wird das IaC Tool Terraform vorgestellt.
Es wird das Einsatzfeld von Terraform im Rahmen von IaC dargestellt
sowie die Funktionsweise von Terraform und die von Terraform
verwendete Sprache erläutert.

Der Hauptteil der Arbeit beschäftigt sich mit der Auswahl der 
Evaluierungskriterien, dem Aufbau des primären Testsystems und
der Definition und Durchführung von zusätzlichen ergänzenden Tests.\\
Die Evaluierungskriterien werden aus dem Qualitätsmodell der ISO
25010 ausgewählt, die individuellen Qualitätsaspekte werden hinsichtlich
ihrere Anwendbarkeit beurteilt und geeignete Aspekte zur Untersuchung
ausgewählt.

Der nächste Teil besteht aus der Vorstellung des primären Testsystems.
Dieses besteht aus drei Terraform-Modulen; dem Root-Modul sowie
zwei nach ihrer Funktionlität getrennten Modulen für einen 
Datenbankserver und einem Kubernetes Cluster.

Dieses System sowie die zusätzlichen Tests werden hinsichtlich der zuvor
ausgewählten Qualitätsmerkmale verglichen und bewertet.

Der letzte Abschnitt der Arbeit umfasst eine Zusammenfassung, 
das abschließende Fazit sowie einen Ausblick auf weitere
untersuchenswerte Themen im Zusammen mit Infrastructure as Code
und Terraform. 

\section{Fazit}

Abschließend lässt sich die Aussage treffen dass sowohl Google Cloud
Plattform als auch Microsoft Azure gut für den Einsatz von Infrastructure
as Code mit Terraform geeignet sind. Beide Terraform Provider befinden
sich in einem ausgereiften Zustand, grundlegend fehlende
Funktionalitäten wurden im Rahmen dieser Evaluierung nicht festgestellt.
Betrachtet man die Details macht Microsoft Azure insgesamt eine
leicht bessere Figur. Die Deployment Zeiten sind im Durchschnitt
etwas schneller und es existieren einige Funktionalitäten die von
Google Cloud Platform aktuell noch nicht vollständig abgedeckt sind.
Die Dokumentation der Terraform Provider beider Cloud Plattformen
fällt sehr gut und umfangreich aus, inklusive einiger Beispiele wie
die individuellen Ressourcen konfiguriert werden können bzw. müssen.

\section{Weitere untersuchenswerte Aspekte und aktuelle Entwicklungen}

Im Verlauf dieser Arbeit wurden eine Vielzahl von Themen angeschnitten
oder erwähnt die für die Umsetzung eines kundenorientierten Projekts
mitunter große Relevanz besitzen, aufgrund des inhaltlichen und
zeitlichen Rahmens jedoch nicht adäquat behandelt werden konnten.\\
Das vielleicht wichtigste dieser Themen ist das Testen von Infrastruktur
Code. Interessant wäre hier nicht nur die Untersuchung der existierenden
Möglichkeiten von Terraform sondern auch der Vergleich mit anderen
Tools wie Pulumi. Allgemein wäre nicht nur der Vergleich der
Testmöglichkeiten interessant, eine grundsätzliche Evaluierung beider
Tools könnte interessante Erkentnisse über die Eignung dieser
in verschiedenen Szenarien liefern und verschiedene Aspekte beider Tools
genauer betrachten.\\
Ein weiteres interessantes Thema ist die Cloud Agnostizität von
Terraform. Das Erfassen des Unterschieds zwischen Erwartung und Realität
sowie das Erforschen der bestehenden Möglichkeiten zum Erreichen einer
\enquote{echten} Cloud Agnostizität in Form einer fundierten Arbeit
wäre mit Sicherheit eine wertvolle Referenz für Neueinsteiger in das
Thema IaC mit Terraform.\\
Im Grundlagenkapitel dieser Arbeit wurde dargestellt warum die
Automatisierung von Infrastruktur von Beginn an essentiell ist um
den vollen Umfang der Vorteile ausschöpfen zu können. Ein Großteil
der aktuell existierenden Projekte ist jedoch aufgrund der
verhältnismäßigen Neuheit von IaC und Terraform mit hoher
Wahrscheinlichkeit noch nicht in dieser Form umgesetzt worden.
Eine Erforschung von Best Practices und Tools zur Automatisierung
bestehender Systeme hätte zum aktuellen Zeitpunkt und auch in
kommenden Jahren einen hohen Wert.\\

\end{spacing}
