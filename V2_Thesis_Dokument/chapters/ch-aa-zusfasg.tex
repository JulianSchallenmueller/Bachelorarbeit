\chapter*{Kurz-Zusammenfassung}

Im Rahmen dieser Arbeit wird ein Vergleich der Cloud Service Provider Microsoft und Google in Bezug auf deren Unterst�tzung von Infrastructure as Code mit Terraform durchgef�hrt.\\
Dieser Vergleich soll die Qualit�t der Unterst�tzung von deren Plattformen Micrsoft Azure und Google Cloud Platform in einem gew�hnlichen Szenario evaluieren und die Frage nach der Einsatzreife beider Plattformen beantworten.\\
Zus�tzlich soll diese Arbeit ermitteln wie einheitlich die konkrete Umsetzung von Infrastructure as Code mit Terraform in einem funktional gleichwertigen Szenario f�r verschiedene Plattformen ausf�llt.

F�r die Beantwortung dieser Fragen wurde ein beispielhaftes Infrastruktur-System auf beiden Plattformen mithilfe von Terraform implementiert und deployed. Die in der Evaluierung eingesetzten Bewertungskriterien wurden aus den Kriterien der des Softwarequalit�tsmodell der ISO/IEC 25010 ausgew�hlt.\\ Zus�tzlich zu der Untersuchung des Testsystems wurde eine Reihe unterst�tzender Versuche durchgef�hrt, durch diese konnte ein Teil der aufgeworfenen Fragen beantwortet und ein vollst�ndigeres Bild bez�glich der Performance, Qualit�t und Unterschiede der untersuchten Plattformen aufgebaut werden.

Das Ergebnis der Untersuchung l�sst darauf schlie�en dass beide Plattformen eine gute Unterst�tzung von Terraform bieten, Azure genie�t hierbei einen leichten Vorsprung hinsichtlich Funktionsumfang und Performance.

Die Einheitlichkeit der Umsetzung f�llt jedoch eher gering aus. Sie begrenzt sich auf die Anwendung einer einheitlichen Sprache und Bedienung durch Terraform, es ist jedoch weiterhin ein umfangreiches Verst�ndnis der individuellen Cloud Plattform notwendig um diese erfolgreich einsetzen zu k�nnen.

 Dennoch ist es sinnvoll Terraform f�r das Deployment von Infrastruktur zu nutzen, durch die Umsetzung der Prinzipien von Infrastructure as Code werden die Vorteile moderner Cloud Plattformen in vollem Umfang ausgesch�pft und die Wertsch�pfungskette von der Idee zur Auslieferung an Kunden kann deutlich beschleunigt werden.�