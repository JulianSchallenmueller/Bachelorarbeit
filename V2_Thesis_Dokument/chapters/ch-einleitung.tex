\chapter{Einleitung}
\label{sec:einl}

Eines der wichtigsten Schlagworte im Zeitalter der fortschreitenden Digitalisierung ist der Begriff des Cloud Computings. Cloud Computing spielt heute l�ngst nicht mehr nur in der IT-Industrie eine wichtige Rolle. Selbst in Bereichen wie der Finanzbranche, die besonders hohen Sicherheitsanspr�chen gerecht werden muss, findet Cloud-basierte Software eine zunehmende Verbreitung \cite{verbr-cloud}.

Die Nutzung von Cloud Technologien verspricht die M�glichkeit
schneller auf Anforderungen von Kunden reagieren zu k�nnen, kosteng�nstige und flexible Skalierung der eigenen Rechenkapazit�ten, Einsparungen durch den Wegfall eigener IT-Infrastruktur-Fachleute und mehr. Gemeinsam mit der Er�ffnung neuer M�glichkeiten bringt die Einf�hrung neuer
Technologie jedoch auch immer eine Reihe eigener Herausforderungen
 mit sich. F�r den erfolgreichen und gewinnbringenden Einsatz dieser Technologie ist es daher essentiell diese zu verstehen und die neuen Herausforderungen mit angepasster Denkweise und neuen Werkzeugen anzugehen.

Das Thema mit dem sich diese Arbeit befasst wird ist das automatisierte
 Management und die Bereitstellung von IT-Infrastruktur-Ressourcen, ein Teil der gr��eren Fachthematik Infrastructure as Code (IaC). Die Grundlagenkapitel werden zu diesem Zweck auf
den technischen Kontext und die Relevanz von Cloud Computing, IaC und
das Software Tool Terraform eingehen.
Es wird erl�utert werden an welcher Stelle die entsprechenden Plattformen
und Software zum Einsatz kommen, welche Probleme durch diese gel�st werden,
wo deren Vorteile und Grenzen liegen sowie welche Alternativen existieren
und wo erg�nzende Werkzeuge zum Einsatz kommen k�nnen.

Den Kern der Arbeit bildet ein Vergleich verschiedener Cloud
Service Provider unter dem zentralen Kriterium derer Unterst�tzung von IaC mit Terraform.
Zu diesem Zweck wurde ein Infrastruktur-System auf Basis des Infrastruktur-Showcase der Firma Novatec\footnote{Link: \url{https://github.com/NovatecConsulting/technologyconsulting-showcase-infrastructure}} f�r verschiedene Cloud Plattformen in Terraform Code implementiert und auf diesen deployed. Als Grundlage f�r den Vergleich dienen eine Auswahl von Aspekten aus dem Softwarequalit�tsmodell der ISO/IEC 25010. Da jedoch nicht alle enthaltenen Qualit�tsaspekte der Norm f�r diesen konkreten Vergleich geeignet sind werden diese zun�chst auf ihre Anwendbarkeit im vorliegenden Fall hin analysiert und bewertet, anschlie�end werden einige  der relevantesten Kriterien ausgew�hlt und die Untersuchung anhand derer durchgef�hrt. Da sich durch Unterschiede zwischen den Cloud Plattformen, zum Beispiel bei Auswahl der Leistungsf�higkeit verwendeter Virtueller Maschinen,  einige Auff�lligkeiten ergeben werden zus�tzliche Tests definiert und durchgef�hrt um ein insgesamt vollst�ndigeres Bild liefern zu k�nnen.\\
Im Anschluss werden die Ergebnisse des Vergleichs bewertet und die daraus resultierenden Erkenntnisse zusammengefasst. Aus diesen Erkenntnissen kann dann ein Fazit gezogen werden das deren Bedeutung im Kontext des aktuellen Stands der Technik interpretiert und bewertet. Den Abschluss bildet ein Ausblick auf zus�tzliche Themen die als n�chstes betrachtet werden sollten wenn es darum geht Infrastructure as Code f�r das Infrastruktur-Management in einer realen Produktivumgebung einzusetzen. Diese Themen werden im Verlauf der Arbeit bereits angesprochen, k�nnen hier jedoch noch nicht in zufriedenstellendem Umfang betrachtet werden.