\chapter{Schluss}
\label{sec:schluss}

\section{Fazit}

Abschlie�end l�sst sich die Aussage treffen dass sowohl Google Cloud
Plattform als auch Microsoft Azure gut f�r den Einsatz von Infrastructure
as Code mit Terraform geeignet sind. Beide Terraform Provider befinden
sich in einem ausgereiften Zustand, das Fehlen grundlegender Funktionalit�ten wurde im Rahmen dieser Evaluierung nicht festgestellt.

Betrachtet man die Details macht Microsoft Azure insgesamt eine
leicht bessere Figur. Die Deployment Zeiten sind im Durchschnitt
etwas schneller und es existieren einige Detail-Funktionalit�ten die von
Google Cloud Platform aktuell noch nicht vollst�ndig abgedeckt sind.
Die Dokumentation der Terraform Provider beider Cloud Plattformen
f�llt sehr gut und umfangreich aus, darin enthaltene Beispiele f�r die Konfiguration
der individuellen Ressourcen erleichtern den Einstieg in Terraform bedeutend.

\section{Ausblick} 

Im Verlauf dieser Arbeit wurden eine Vielzahl von Themen angeschnitten
oder erw�hnt die f�r die Umsetzung eines kundenorientierten Projekts
mitunter gro�e Relevanz besitzen, aufgrund des inhaltlichen und
zeitlichen Rahmens jedoch nicht ad�quat behandelt werden konnten.\\
Das vielleicht wichtigste dieser Themen ist das Testen von Infrastruktur
Code. Interessant w�re hier nicht nur die Untersuchung der existierenden
M�glichkeiten von Terraform sondern auch der Vergleich mit anderen
Tools wie Pulumi. Allgemein w�re nicht nur der Vergleich der
Testm�glichkeiten interessant, eine grunds�tzliche Evaluierung beider
Tools k�nnte interessante Erkenntnisse �ber die Eignung dieser
in verschiedenen Szenarien liefern und verschiedene Aspekte beider Tools
genauer betrachten.\\
Ein weiteres interessantes Thema ist die Cloud Agnostizit�t von
Terraform. Das Erfassen des Unterschieds zwischen Erwartung und Realit�t
sowie das Erforschen der bestehenden M�glichkeiten zum Erreichen einer
\enquote{echten} Cloud Agnostizit�t in Form einer fundierten Arbeit
w�re mit Sicherheit eine wertvolle Referenz f�r Neueinsteiger in das
Thema IaC mit Terraform.\\
Im Grundlagenkapitel dieser Arbeit wurde dargestellt warum die
Automatisierung von Infrastruktur von Beginn an essentiell ist um
den vollen Umfang der Vorteile aussch�pfen zu k�nnen. Ein Gro�teil
der aktuell existierenden Projekte ist jedoch aufgrund der
verh�ltnism��igen Neuheit von IaC und Terraform mit hoher
Wahrscheinlichkeit noch nicht in dieser Form umgesetzt worden.
Eine Erforschung von Best Practices und Tools zur Automatisierung
bestehender Systeme h�tte zum aktuellen Zeitpunkt und auch in
kommenden Jahren einen hohen Wert.\\