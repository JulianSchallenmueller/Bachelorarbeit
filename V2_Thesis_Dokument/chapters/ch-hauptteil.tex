% Bsp. eines Hauptteils

\chapter{Grundlagen}
\label{sec:grundl}

\section{Funktionsprinzip, Vorteile und Herausforderungen des modernen Cloud Computings}

\subsection{Definition und Funktionsweise}

\subsection{Vor- und Nachteile im Einsatz von Cloud Computing}

\subsection{�berblick �ber die wichtigsten Public Cloud Service Provider}

\section{Stand der Technik}

\subsection{Eingesetzte Technik zur Realisierung eines Konzepts als Cloud-basierte Software}



\section{Infrastructure as Code}

\subsection{Technologischer Wandel und das Cloud Age Mindset}

\subsection{Vorteile von Infrastructure as Code im Vergleich zu manuellem Infrastruktur-Provisioning}

\subsection{Herausforderungen im Einsatz von Infrastructure as Code}

\subsection{Die drei Kernverfahren von Infrastructure as Code}

\section{Funktionsprinzip und Rolle von Terraform im IaC-Anwendungsprozess}

\subsection{Funktionsweise von Terraform}

\subsection{�berblick �ber die Hashicorp Configuration Language}




\chapter{Evaluierungsanforderungen und Umsetzung}
\label{sec:ergeb}

\section{Evaluierungsanforderungen}

\subsection{Ziel der Evaluierung}

\subsection{Untersuchte Komponenten der Terraform Provider}

\subsection{Auswahl der Evaluierungskriterien}




\section{Umsetzung des Testsystems}

\subsection{Eingesetzte Software und Tools}

\subsection{High-Level Aufbau des Testsystems}

\subsection{Konkreter Aufbau auf Google Cloud Platform}

\subsection{Konkreter Aufbau auf Azure}

\subsection{Aufbau der erg�nzenden Versuche}




\chapter{Ergebnisse und Bewertung}

\section{Evaluierung der Functional Completeness}

\section{Ergebnisse und Bewertung der Time Behaviour Tests}

\section{Ergebnisse und Bewertung der Recoverability Tests}

\section{Ergebnisse und Bewertung der Modifiability Tests}

\section{Evaluierung der Einheitlichkeit der Testsysteme f�r Azure und Google Cloud Platform}




